%!TEX encoding = UTF-8 Unicode
\documentclass[11pt, a4paper, twoside]{article}   	% use "amsart" instead of "article" for AMSLaTeX format

\usepackage{geometry}                		% See geometry.pdf to learn the layout options. There are lots.
\usepackage{listings}				% For Source Code displaying
\usepackage[german]{babel}			% this end the next are needed for german umlaute
\usepackage[utf8]{inputenc}
\usepackage{color}
\usepackage{graphicx}
\usepackage{titlesec}
\usepackage{fancyhdr}
\usepackage{lastpage}
\usepackage{hyperref}
% http://www.artofproblemsolving.com/wiki/index.php/LaTeX:Symbols#Operators
% =============================================
% Layout & Colors
% =============================================
\geometry{
   a4paper,
   total={210mm,297mm},
   left=20mm,
   right=20mm,
   top=20mm,
   bottom=30mm
 }	

\definecolor{myred}{rgb}{0.8,0,0}
\definecolor{mygreen}{rgb}{0,0.6,0}
\definecolor{mygray}{rgb}{0.5,0.5,0.5}
\definecolor{mymauve}{rgb}{0.58,0,0.82}

\setcounter{secnumdepth}{4}


% the default java directory structure and the main packages
\newcommand{\srcDir}{../src/main/java}
\newcommand{\srcTestDir}{../src/test/java}
\newcommand{\mainPackage}{\srcDir/at/fhooe/swe4/lab3}
\newcommand{\mainTestPackage}{\srcTestDir/at/fhooe/swe4/lab3/test}
\newcommand{\junitReportDir}{junit-report}
% the default subsection headers
\newcommand{\ideaSection}{Lösungsidee}
\newcommand{\sourceSection}{Source-Code}
\newcommand{\testSection}{Tests}


% =============================================
% Code Settings
% =============================================
\lstset{ %
  basicstyle=\tiny,        % the size of the fonts that are used for the code
  breakatwhitespace=false,         % sets if automatic breaks should only happen at whitespace
  breaklines=true,                 % sets automatic line breaking
  captionpos=t,                    % sets the caption-position to top
  commentstyle=\color{mygreen},    % comment style
  frame=single,                    % adds a frame around the code
  keepspaces=true,                 % keeps spaces in text, useful for keeping indentation of code (possibly needs columns=flexible)
  keywordstyle=\color{blue},       % keyword style
  language=JAVA, 
  numbers=left,                    % where to put the line-numbers; possible values are (none, left, right)
  numbersep=5pt,                   % how far the line-numbers are from the code
  numberstyle=\tiny\color{mygray}, % the style that is used for the line-numbers
  rulecolor=\color{white},         % if not set, the frame-color may be changed on line-breaks within not-black text (e.g. comments (green here))
  showspaces=false,                % show spaces everywhere adding particular underscores; it overrides 'showstringspaces'
  showstringspaces=false,          % underline spaces within strings only
  showtabs=false,                  % show tabs within strings adding particular underscores
  stepnumber=1,                    % the step between two line-numbers. If it's 1, each line will be numbered
  stringstyle=\color{mymauve},     % string literal style
  tabsize=2,                       % sets default tabsize to 2 spaces
  title=\lstname                   % show the filename of files included with \lstinputlisting; also try caption instead of title
}

% =============================================
% Page Style, Footers & Headers, Title
% =============================================
\title{Übung 3}
\author{Thomas Herzog}

\lhead{Übung 3}
\chead{}
\rhead{\includegraphics[scale=0.10]{FHO_Logo_Students.jpg}}

\lfoot{S1310307011}
\cfoot{}
\rfoot{ \thepage / \pageref{LastPage} }
\renewcommand{\footrulewidth}{0.4pt}
% =============================================
% D O C U M E N T     C O N T E N T
% =============================================
\pagestyle{fancy}
\begin{document}
\setlength{\headheight}{15mm}
% =============================================
% Solution Idea
% =============================================
{\color{myred}
	\section
		{Hammingfolge}
}

\subsection{\ideaSection}
Folgend ist die Lösungsidee für die Aufgabenstellung Hammingfolge berechnen angeführt.\\
Da es sich hierbei lediglich um einen einzigen Algorithmus handelt soll dieser als Klassenmethode implementiert werden. Das diese Klasse lediglich diese Klasenmethode enthalten soll, soll in dieser Klasse ein Privater Konstruktor implementiert werden um zu verhindern, dass diese Klasse instanziert werden kann.\\\\
Da eine Hammingfolge wie folgt definiert ist:\\
$1 \in H$ \\
$x \in H \Rightarrow 2 \ast x \in H \wedge 3 \ast x \in H \wedge 5 \ast x \in H$\\
wissen wir dass folgende Elemente Aufgrund dessen das $1 \in H$ gilt in der Folge vorhanden sind. \\
$1 \in H \wedge 2 \in H \wedge 3 \in H \wedge 5 \in H$\\
daher können wir einen Algorithmus definieren der sich wie folgt verhalten soll:
\begin{enumerate}
	\item Instanziere eine NavigableSet\textless E\textgreater und initialisiere dieses Set mit dem Element 1
	\item Instanziere eine ArrayList\textless E\textgreater welches die resultierenden Werte beinhaltet
	\item Polle und entferne das erste Element aus dem Set
	\item Füge dieses Element der resultierenden Liste hinzu.		
	\item Berechne die nachfolgenden Hammingzahlen $(2 \ast polledValue \wedge 3 \ast polledValue \wedge 5 \ast polledValue)$ für dieses Element
	\item Füge die Berechneten Elemente dem Set hinzu
	\item Wiederhole Schritt 3 solange folgendes gilt: $list.size(i) < (n + 4)$
\end{enumerate}
Es soll eine TreeSet\textless E\textgreater Instanz verwendet werden. Es soll aber gegen NavigableSet\textless E\textgreater Interface und nicht SortedSet\textless E\textgreater gearbeitet werden da dieses Interface eine Methode namens pollFirst() zur Verfügung stellt, die das erste Element des Set liefert und es aus dem Set entfernt. Dadurch sollte das Set in seiner Größe beschränkt sein, was den Sortierungsaufwand des Set minimal halten sollte.\\
Da TreeSet aber auch SortedSet implementiert sind die enthaltenen Werte implizit immer sortiert und dadurch auch die Werte in der resultierenden Liste, da die hinzugefügten Elemente immer sortiert eingefügt werden. Daher ist hier kein zusätzlicher Sortierungsaufwand nötig.
\newpage
\subsection{\sourceSection}
Folgend ist der Implementierte Source und Test-Source angeführt.
\lstinputlisting{\mainPackage/hamming/Hamming.java}
\lstinputlisting{\mainTestPackage/hamming/HammingTest.java}
\subsection{\testSection}
Folgend sind die Tests der Aufgabenstellung Hammingfolge angeführt.\\
Aufgrund dessen das JUnit verwendet wurde und JUnit auch eine Report generiert wird hier auf das Einfügen der Tests verzichtet und nur der generierte JUnit Report verlinkt.\\\\
{\color{myred} ACHTUNG: Da der Report mit einen relativen Pfad eingebunden wurde darf das Dokument nicht verschoben werden ohne das gewährleistet ist, dass das Verzeichnis "\junitReportDir", welches die JUnit Reports enthält, wieder relativ gesehen an derselben Position ist}\\\\
JUnit Report öffnen (\href{\junitReportDir/index.html}{index.html})

\newpage
{\color{myred}
	\section
		{Sortieralgorithmen}
}
\subsection{\ideaSection \hspace{2mm}(Allgemein)}
Folgend sind die Lösungsideen der Sortierlagorithmen HeapSorter und QuickSorter angeführt.\\
Da beide Algorithmen denselben output liefern sollen, soll hier ein Interface spezifiziert werden welches die Funktionalität bzw. die zu implementierenden Methoden Signaturen vorgibt. Die Aufgabenstellung verlangt zwar das Sortieren auf Integer Felder, jedoch sollen die Algorithmen so implementiert werden, dass sie auf Typen, die das Interface Compareable<E> implementieren. Daher muss das Interface folgende Signatur vorweisen.
\begin{lstlisting}
public Sorter<E extends Comparable<E>> {...}
\end{lstlisting}
\newpage
\subsubsection{Source Code}
\lstinputlisting{\mainPackage/sort/api/Sorter.java}
\newpage
\subsection{\ideaSection \hspace{2mm}(Statistics)}
Aufgrund dessen das die Sortieralgorithmen mit Code Statistics versehen werden sollen, sollen Klassen implementiert werden, die es erlauben die verlangten Statistics zu ermitteln und auch einen Report dieser zu erstellen.\\
Hierbei soll diese Code Statistics wie folgt aufgeteilt werden:
\begin{enumerate}
	\item \textbf{StatisticsProvider:} Das Interface welches die Spezifikation für den Code Statistics Provider enthalten soll.\\
	Die Implementierung soll es ermöglichen mehrere Statistic Kontexte zu verwalten.
	\item \textbf{StatisticContext:} Die Klasse, welche einen Statstic Kontext darstellt soll.\\
	Dieser Kontext soll es ermöglichen mehrere CodeStatistic Instanzen pro Kontext zu verwalten.
	\item \textbf{CodeStatistic:} Die Klasse, die die Code Statistic Informationen (swap, compare counts) halten soll
	\item \textbf{DefaultStatisticProviderImpl:} Die default Implementierung des Interface StatisticProvider, welches die Funktionalitäten implementiert soll.
\end{enumerate}
Alle Klassen sollen die toString Methode überschreiben und jeweils ihre beinhaltenden Informationen als string zurückliefern, wobei ein Parent bzw. die Klasse die Isntanzen einer anderen verwaltet and deren toString() zu delegieren.
\newpage
\subsubsection{Source Code}
Folgend ist der Source der Statistics Implementierungen und Interfaces angeführt.
\lstinputlisting{\mainPackage/stat/api/StatisticsProvider.java}
\lstinputlisting{\mainPackage/stat/StatisticContext.java}
\lstinputlisting{\mainPackage/stat/CodeStatistics.java}
\lstinputlisting{\mainPackage/stat/DefaultStatisticsProviderImpl.java}

\newpage
\subsection{HeapSorter}
Folgend ist die Lösungsidee für die HeapSorter Implementierung angeführt.\\
Da hierbei eine Heap Implementierung von Nöten ist und diese aber auch anderweitig verwendet werden könnte soll ein Heap Implementiert werden, der unabhängig von einem HeapSorter verwendet werden kann. Da wir auch hier generisch bleiben sollen und es auch möglich sein soll eine Heap Implementierung mit einem anderen Container zu implementieren (Bsp.: ArrayList\textless E\textgreater, T[], usw.) soll ein Interface spezifiziert werden, welches die Funktionalitäten eines Heap spezifiziert. Es soll folgende Signatur haben
\begin{lstlisting}
public Heap<E extends Comparable<E>> {...}
\end{lstlisting}
Des Weiteren soll eine Enumeration spezifiziert werden, die es erlaubt zu definieren, ob der Heap ein upheap oder downheap sein soll, also ob der root das höchste oder kleinste Element ist.\\\\
Ansonsten soll der Heap wie bekannt implementiert werden.
\subsubsection{Source Code}
Folgend ist der Source der Interfaces und Implementierungen für Heap und Heap Sorter angeführt.
\lstinputlisting{\mainPackage/sort/api/Heap.java}
\lstinputlisting{\mainPackage/sort/heap/impl/HeapArrayListImpl.java}
\lstinputlisting{\mainPackage/sort/heap/impl/HeapSorter.java}
\subsubsection{\testSection}
see junit index.html

\end{document}  